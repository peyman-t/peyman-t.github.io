\documentclass{ieeeaccess}
\usepackage{cite}
\usepackage{amsmath,amssymb,amsfonts}
\usepackage{algorithmic}
\usepackage{balance}
\usepackage{graphicx}
\usepackage{textcomp}
\usepackage{enumitem}
\usepackage[hyphens]{url}
\usepackage{hyperref}
\newcommand{\subscript}[2]{$#1 _ {#2}$}
\def\BibTeX{{\rm B\kern-.05em{\sc i\kern-.025em b}\kern-.08em
    T\kern-.1667em\lower.7ex\hbox{E}\kern-.125emX}}
\begin{document}
\history{Date of publication xxxx 00, 0000, date of current version xxxx 00, 0000.}
\doi{10.1109/ACCESS.2023.0322000}

\title{Reconciling Efficiency and Security of the Internet of Things: A Recursive InterNetwork Architecture (RINA) Approach}
\author{\uppercase{Peyman Teymoori}\authorrefmark{1,3},
\uppercase{Toktam Ramezanifarkhani}\authorrefmark{2,3}}

\address[1]{School of Business, University of South-Eastern Norway (e-mail: peyman.teymoori@usn.no)}
\address[2]{School of Economics, Innovation and Technology, Kristiania University College, Norway (e-mail: toktam.ramezanifarkhani@kristiania.no)}
\address[3]{Department of Informatics, University of Oslo, Norway (e-mail: \{peymant | toktamr\}@ifi.uio.no)}
\tfootnote{.}

\markboth
{Peyman Teymoori \headeretal: Reconciling Efficiency and Security of the IoT: A RINA Approach}
{Author \headeretal: Reconciling Efficiency and Security of the IoT: A RINA Approach}

\corresp{Corresponding author: Peyman Teymoori (e-mail: peyman.teymoori@usn.no).}


\begin{abstract}
%The Internet of Things (IoT) has transformed our lives by connecting small devices to the internet, enabling automation and simplifying our daily routines. However, this advancement has brought forth significant challenges, particularly at the communication and networking layer of IoT. One pressing research question has been the ability to achieve both transport layer efficiency and security simultaneously—a complex problem that has remained open. In this paper, we delve into this challenge by providing a comprehensive overview of the current network stacks used in IoT: IP- and ICN-based network stacks. Elaborating current issues and drawing upon real-world use cases, we showcase how the concept of recursion in networking can address this predicament. Our architectural approach leverages the Recursive InterNetwork Architecture (RINA), which exemplifies how the fusion of recursion and networking can resolve the efficiency-security trade-off inherent in IoT networks. Through recurring the same type of layer but with different scopes, we highlight how RINA ensures security and efficiency concurrently; \textit{this inherently allows for securely breaking an end-to-end connection to increase its performance}; This transcends the limitations of existing architectures and solves the long-standing dilemma with Performance Enhancing Proxies (PEPs) in the Internet design that cannot operate on encrypted connections. Our approach enables the realization of secure multicast, paving the way for a more robust and reliable IoT ecosystem, capable of meeting the evolving demands of our connected world.

The Internet of Things (IoT) has revolutionized our lives by connecting devices to the internet, enabling automation and simplifying daily routines. However, as IoT is built upon the foundation of older network architectures and protocols, such as the Internet, its integration has brought forth significant challenges, particularly in achieving both transport layer efficiency and security at the same time. This paper presents a comprehensive overview of the current network stacks used in IoT and highlights the issues they face with this regards. We propose a novel architectural approach leveraging the Recursive InterNetwork Architecture (RINA) to address these challenges. RINA's unique combination of recursion and networking can effectively reconcile the efficiency-security trade-off inherent in IoT networks, overcoming the limitations of existing architectures and resolving the long-standing issue with Performance Enhancing Proxies (PEPs) that cannot operate on encrypted connections. We demonstrate the practical application of RINA through two use cases: a smart home environment and healthcare patient monitoring. The paper concludes with a comprehensive discussion on potential future research topics, paving the way for an efficient and secure IoT ecosystem.
\end{abstract}

\begin{keywords}
Efficiency, Internet of Things, Multicast, Recursive InterNetwork Architecture, Security.
\end{keywords}

\titlepgskip=-21pt

\maketitle

\section{Introduction}
\label{sec:intro}
The IoT aims to provide a communication network infrastructure with inter-operable protocols and software for physical/virtual sensors, personal computers (PCs), smart devices, automobiles, and other items, such as a refrigerator, dishwasher, microwave oven, food, and medicines, anytime and on any network. Since IoT embraces many different technologies, services, and standards, it is widely perceived as a main pillar of the ICT market in the next ten years \cite{Alaba201710}. Many new applications such as health-care (e.g. remote patient/elderly people monitoring) and home automation (e.g. heating and lightning control) have been developed so far. This has led to efforts conducted by standardization bodies such as the Institute of Electrical and Electronics Engineers (IEEE) and the Internet Engineering Task Force (IETF) \cite{rfc8576}, towards the design of communication and security technologies for the IoT. Given the availability of wireless communication technologies, the number of developed things is expected to rapidly increase: this also raises serious challenges: the network can be easily targeted by attacks and malicious users. What IoT needs is a well-established \textit{efficient} communication platform across different domains, capable of ensuring the \textit{security} of information.

%Although most of these issues have been found and tried to be addressed before in the Internet, however, IoT still faces almost the same issues; to improve the efficiency at the transport layer, performance enhancement methods such as PEPs \cite{caini2007pepsal} and CoAP gateways \cite{rfc7252} have been proposed in IoT. DTLS, which is based on TLS but operating on datagrams, has been adopted by IoT to address its security challenges \cite{rfc6347}. Many IoT protocols such as MQTT \cite{mqtt2014} and CoAP utilize TLS or DTLS. However, the performance of CoAP proxies is affected inversely by adding security features to other layers especially, transport. 
While the Internet has faced and tried to address many similar challenges, the Internet of Things (IoT) continues to grapple with comparable issues. This is primarily due to the unique characteristics and requirements of IoT networks. To enhance the efficiency at the transport layer, several performance enhancement methods have been proposed, such as Performance Enhancing Proxies (PEPs) \cite{caini2007pepsal} and Constrained Application Protocol (CoAP) gateways \cite{rfc7252}.

To tackle the security challenges inherent in IoT, Datagram Transport Layer Security (DTLS) \cite{rfc6347}, a protocol based on Transport Layer Security (TLS) but designed to operate on datagrams, has been widely adopted. Many IoT protocols, including Message Queuing Telemetry Transport (MQTT) \cite{mqtt2014} and CoAP, utilize either TLS or DTLS to ensure secure communication.

However, it is important to note that the addition of security features, particularly at the transport layer, can inversely affect the performance of CoAP proxies. The main reasons include \cite{grammatikis2019securing}
\begin{enumerate}
	\item limitations of the DTLS Handshake Protocol with large messages,
	\item complications of using DTLS with CoAP, and
	\item not supporting multicast communications by DTLS.
\end{enumerate}

In other words, adding (security) functionalities to one layer might interfere proper (and mostly performance) operations at the other layers \cite{7005393}. The problem worsens when an IoT application is supposed to operate over multiple cross-domain nodes and heterogeneous environments; this is the cause of diversity of protocols and policies and their communication, especially from the security point of view. %thus vulnerabilities and applicable attacks

In this paper, we propose a novel architectural approach leveraging RINA to address these challenges. We argue that RINA's unique combination of recursion and networking can effectively reconcile the efficiency-security trade-off inherent in IoT networks, overcoming the limitations of existing architectures and resolving the long-standing issue with Performance Enhancing Proxies (PEPs) that cannot operate on encrypted connections.

%In this paper, we show that the contradiction between transport layer efficiency and security arises from the architectural deficiency of IoT, and to demonstrate how this problem can be solved, we present an architectural solution based on the idea of recursiveness in networking functionality. In particular, we focus on the Recursive InterNetwork Architecture (RINA), and show how this architecture can meet both efficiency and security, and how recursiveness can facilitate providing a unified interoperable and programmable architecture. 

RINA was first introduced in \cite{Day:2008:PNA} and then, implemented and evaluated by a number of international projects. Recursion in RINA is performed by defining a layer as a foundation with basic \textit{mechanisms} to provide Inter Process Communication (IPC) between two IPC Processes (IPCP). This layer is called Distributed IPC Facility (DIF). DIFs (the \textit{mechanisms} inside DIFs) can be customized through adopting different \textit{policies}. Then, DIFs can be recursively arranged to form different topologies \cite{peymanICC16}. 
%The recursion idea and using fixed mechanisms at every layer reduces the number of protocols and security challenges at different network layers since the same code/program is instantiated. The recursive idea of RINA enables us to solve the aforementioned problems of IoT regarding its transport efficiency and security.

%RINA provides inter-DIF security through its Service Data Unit (SDU) Protection module. This module is responsible for ensuring the integrity and confidentiality of data as it transits between DIFs. The separation of intra-DIF and inter-DIF security in RINA facilitates protocol translation and connection splitting along the communication path without sacrificing security.

RINA's approach to transport layer security is particularly advantageous. By separating security concerns at different layers, embedding security within each layer of the communication stack, and arbitrary stacking these layers, RINA effectively reconciles the traditional trade-off between security and efficiency. This is also confirmed by RINA's inherent support for secure multicast, a critical feature for IoT networks that is often overlooked in conventional network protocols.

To validate RINA's benefits to IoT, the paper also presents two distinct use cases: a smart home environment and healthcare patient monitoring. The first use case, the smart home, comprises various devices from different vendors, each with its own operational and security requirements. 
%The application of RINA provides a unified networking framework, reducing the complexities of integrating these diverse systems. Using RINA's structure, it establishes a secure and efficient network that enhances interoperability between IoT devices and cloud servers, despite their different protocols or standards. The architecture also enables customization of congestion control policies and Service Data Unit (SDU) protection policies for each path segment, thereby enhancing the overall network security and efficiency.
The second use case focuses on the healthcare sector, where IoT devices have become crucial in patient monitoring and care. In this scenario, wearable IoT devices, which might not inherently support RINA, are incorporated into a RINA network using an IoT sub-manager, a special type of gateway. %This setup ensures seamless integration of legacy devices while maintaining the advantages of the RINA architecture, including security and efficient data transmission. The IoT sub-manager translates the legacy protocol/packets from the IoT device into a format compatible with the RINA network.

The primary contributions of this paper are:
\begin{itemize}
	\item A comprehensive overview of the current network stacks used in IoT and the issues they face.
	\item Presenting a novel architectural approach that leverages RINA to address significant challenges in IoT, particularly in achieving both transport layer efficiency and security, and offering practical examples of its application through two use cases.
	\item A comprehensive discussion on potential future research topics, identifying areas where further exploration could yield significant insights.
\end{itemize}


%These use cases demonstrate how RINA can effectively address the key challenges of security, interoperability, and efficient resource usage in IoT networks, providing a more seamless and secure experience in both smart home and healthcare settings.

%This paper is organized as follow: beginning with Section \ref{sec:iot}, we delve into the current security challenges confronting IoT, particularly within its transport layer. We transition in Section \ref{sec:rina} to a detailed overview of RINA, emphasizing its transport features and their relevance to IoT. Sections \ref{sec:rina-trans} and \ref{sec:rina-other} further illustrate how RINA could effectively address these IoT challenges. In Section \ref{sec:rina-tradeoff}, we explore RINA's unique features that can mitigate the persistent trade-off between efficiency and security in IoT. Section \ref{sec:usecases} brings our discussion into real-world application, presenting three distinct use cases where RINA can be effectively employed in IoT environments. Finally, we discuss future research topics in Section \ref{sec:disc} and summarize our findings and conclusions in Section \ref{sec:conclusion}.

The remainder of this paper is organized as follows: 
In Section \ref{sec:iot}, we delve into the current security challenges confronting IoT, particularly within its transport layer, providing a comprehensive overview of the issues at hand. Section \ref{sec:rina} introduces RINA, detailing its transport features and their relevance to IoT. Sections \ref{sec:rina-trans} and \ref{sec:rina-other} illustrate how RINA could effectively address these IoT challenges, providing a thorough analysis of its potential benefits. 

In Section \ref{sec:rina-tradeoff}, we explore RINA’s unique features that can mitigate the persistent trade-off between efficiency and security in IoT, presenting a compelling argument for its adoption. Section \ref{sec:usecases} presents three distinct use cases where RINA can be effectively employed in IoT environments, offering practical examples of its application. In Section VIII, we discuss potential future research topics, identifying areas where further exploration could yield significant insights. Finally, Section \ref{sec:conclusion} summarizes our findings and conclusions, encapsulating the key points of our discussion and highlighting the potential of RINA in the context of IoT.

\section{Issues of Current IoT Network Stacks} \label{sec:iot}
What IoT needs is an efficient communication platform across different domains, which is capable of ensuring security. We argue that IoT security challenges mostly stem from the architectural design of the IoT network stack. There are mainly two tracks of efforts on improving the security in IoT. One of them utilizes the current Internet design (i.e. TCP/IP) as the base, and the other adopts new frameworks such as Information-Centric Networking (ICN).  



\begin{figure} 
	\centering
	\includegraphics[width=0.17\textwidth]{figures/IoTStack.pdf}
	\caption{A typical IoT network stack with common protocols.}
	\label{fig:IoTStack}
\end{figure}


\subsection{IP-Based Stack}
As IoT envisions a future Internet in which everyday objects possessing sensing and actuating capabilities cooperate with computer systems, IP-based communication protocols have been aligned with IoT devices to form a communications stack. 
Fig.~\ref{fig:IoTStack} illustrates a typical network stack of IoT with common protocols in each layer; up to the MAC layer, IEEE 802.15.4 \cite{6012487} is usually supported; the 6LoWPAN protocol \cite{rfc4919} enables the transmission of IPv6 over IEEE 802.15.4; RPL \cite{rfc6282} provides routing over 6LoWPAN; TLS \cite{rfc5246} and DTLS \cite{rfc6347} represent transport layer security using TCP and UDP, respectively; and CoAP \cite{rfc7252} provides web transfer at the application layer over UDP (DTLS) with some limited congestion control features. Since we focus on the network stack of IoT devices, we do not discuss \textit{perception} (to perceive the environment with technologies such as RFIS and GPS) which is usually categorized as a layer.

A typical network stack of IoT with common protocols in each layer resembles the TCP/IP stack with some commonalities, especially in the design. However, the large-scale deployment of IP-based IoT solutions is still challenging \cite{amadeo2016information}. 

In \cite{ramezanifarkhani2018securing}, it has shown that IoT security challenges of IP-based IoT frameworks mostly stem from the architectural design of the IoT network stack. In addition, comprehensive studies such as \cite{yang2017survey} on IoT security have confirmed that many issues stem from the currently-employed network stack protocols of IoT devices and their interoperability issues with the Internet. There are some major flaws with IP: the concept of scope of a layer is not used correctly \cite{day2008networking}, as it has been diluted by gradual updates. Although some research work (e.g. \cite{suarez2016secure}) focused on presenting a secure IoT architecture, it usually operates at higher layers regardless of what the network stack is. On the contrary, in this paper, we fundamentally look at the lower layers, and especially the network stack and its protocols.

At the transport/application layer, DTLS, as the common protocol, have been in use to ensure end-to-end security using CoAP. As its limitations were mentioned by \cite{grammatikis2019securing,7005393}, DTLS does not support multicast communication which is highly required in IoT networks. Due to its end-to-end security, DTLS also complicates operations of CoAP proxies in the path. This problem has led to securing CoAP communications and object security rather than the transport security provided by DTLS. However, this approach is not mature yet.
Although most of these issues have been found and tried to be addressed before in the Internet, however, IoT still faces almost the same issues.


\subsection{ICN-Based Stack}
%Different from the IP network, Information-Centric Networking (ICN) assigns unique names to content, regardless of its location (e.g. IP address) and generating application, which enables in-network caching/replication and content-based security \cite{amadeo2016information}. Although the early ICN architects did not have IoT in mind, ICN has been adopted to address IoT requirements. However, the uniqueness and complexity of IoT requirements raise challenges to the design of ICN \cite{amadeo2016information}. Despite the standardization efforts on ICN-IoT deployments within the ICN Research Group (ICNRG) \cite{irtf-icnrg-icniot-03}, it has been under debate in the ICN community that ICN can face serious security and performance issues in IoT \cite{amadeo2016information,fang2018survey} since ICN-IoT not only connects many more devices than most other networks, but it also assigns (unique) names to their content. 
Contrary to IP networks, Information-Centric Networking (ICN) adopts a unique approach by assigning distinct identifiers to content, independent of its originating location or application. This approach facilitates in-network caching/replication and content-based security \cite{amadeo2016information}. Although the original architects of ICN did not specifically design it with IoT in mind, ICN has been repurposed and adapted to cater to the unique requirements of IoT.

However, the multifaceted and complex nature of IoT demands presents unique challenges to the design and implementation of ICN \cite{amadeo2016information}. Even with ongoing standardization efforts for ICN-IoT deployments within the ICN Research Group (ICNRG) \cite{irtf-icnrg-icniot-03}, there exists a considerable debate within the ICN community. The concerns center around potential security and performance issues that ICN could encounter in the IoT context \cite{amadeo2016information,fang2018survey}.

This is primarily due to the fact that ICN-IoT not only connects a significantly higher number of devices compared to most other networks, but it also assigns unique names to the content generated by these devices. This combination of vast connectivity and content-specific naming imposes considerable challenges in terms of managing security protocols and optimizing performance in an IoT environment. 

\subsection{Summary of IoT Networking Challenges} \label{sec:from-prev}

Although there has been much research work on securing IoT, there are still open issues regarding the layers, protocols, and their vulnerabilities; these issues are thoroughly discussed by \cite{7005393, yang2017survey}. 
% 
%concluding from the above issues which are also the results of recent surveys on the security of IoT networks (e.g. \cite{7005393, yang2017survey}),
We argue that IoT security challenges mostly stem from the architectural design of IoT. The challenges are as follow\footnote{These challenges were originally published in \cite{ramezanifarkhani2018securing} in a much shorter form.}:

\subsubsection{IoT Network Stack Challenges}
Despite the lessons learned from the Internet, the IoT network stack has similar security issues.	 
Referring to Fig.~\ref{fig:IoTStack}, the 6LoWPAN protocol does not define any security mechanism, but it makes the use of IPsec available to provide security between two communication end points. However, no specific method has been adopted yet for 6LoWPAN \cite{7005393}. Since the 6LoWPAN Border Router typically does not perform any authentication, IoT networks are still vulnerable \cite{yang2017survey}. Moreover, encryption at the routing layer hides all the necessary information of the upper layers; this is one of the main problems that Performance Enhancing Proxies (PEPs) \cite{caini2007pepsal} are facing in wireless networks because they cannot break the end-to-end congestion control loop to start a new one matching the environment properties (e.g. wired, wireless) \cite{thai2011satern}. Although there have been proposals on smart gateways (e.g. \cite{bimschas2010middleware,7931688}) to connect \textit{things} to the Internet, the same problem still exists \cite{7005393}.

At the routing layer, RPL \cite{rfc6550} is commonly used, which does not define how to protect RPL communications and operations from internal attackers, and it also lacks some key security features \cite{7005393}. 

At the transport/application layer, DTLS, as the common protocol, have been in use to ensure end-to-end security using CoAP. As its limitations were mentioned by \cite{7005393}, DTLS does not support multicast communication which is highly required in IoT networks. Due to its end-to-end security, DTLS also complicates operations of CoAP proxies in the path. This problem has led to securing CoAP communications and object security rather than the transport security provided by DTLS. However, this approach is not mature yet.

\subsubsection{Repeated Functionality in Layers/Protocols}
%As we see, almost the same (security) functionality needs to be repeated/employed/implemented at different layers several times per protocol while there is ``no need to reinvent the wheel" in several layers. This repetition increases the risk of making new mistakes/vulnerabilities, and reduces the reuse degree for future extensions.
As observed, similar security functionalities are often redundantly implemented across different layers within multiple protocols. This repetition not only reflects an inefficient use of resources but also an unnecessary complexity that contradicts the principle of ``no need to reinvent the wheel'' across several layers.

Each time a function is reimplemented, there's a risk of introducing new vulnerabilities into the system. This is particularly relevant in the context of security functions, where a minor oversight can result in significant security breaches. Furthermore, the multiplication of similar functions in different layers often complicates the process of patching and updating security protocols, thereby increasing the system's vulnerability to threats over time.

Moreover, this redundancy impedes the degree of reusability for future extensions. A more modular design, where functions are implemented as independent, reusable modules, could significantly enhance the scalability and adaptability of the system. This modularity not only allows for more efficient resource utilization but also enables faster development and deployment of new functionalities.

\subsubsection{Global, Public, and Large Address Space}
%Due to the overwhelming number of \textit{things}, IPv6 was employed. It also comes with its own challenges such as large, public addresses since each \textit{thing} should be addressable publicly. Knowing target addresses indeed increases security challenges in IoT. Since the IP addresses are globally public, adopting them is complicated due to their large size, while there is no need to expose everything to the widest scope, i.e. the Internet. 
Given the overwhelming number of IoT devices, or ``things'', IPv6 was adopted to cater to the expansive address space requirement. However, the implementation of IPv6 introduces its own set of challenges, one of the most significant being the management of large, public addresses \cite{das2023recent}.

In an IPv6 environment, every IoT device requires a unique, globally addressable IP, which inherently increases the exposure and vulnerability of these devices. This high visibility leads to heightened security risks as potential attackers can easily identify and target specific devices. While providing each device with a unique, publicly accessible address facilitates communication, it simultaneously unveils a plethora of devices to potential cyber threats.

Furthermore, the adoption of IPv6 addresses presents practical difficulties due to their large size. Managing and maintaining such an extensive addressing space necessitates robust and efficient networking infrastructure, which can be resource-intensive and challenging to implement, particularly on a global scale.

Moreover, the current approach of exposing every device to the widest scope -- the Internet -- isn't always necessary or ideal. Not every IoT device needs to be globally addressable; many can function effectively within local or private networks. Employing a more nuanced, scope-dependent addressing strategy can significantly mitigate the security risks associated with global, public addressing.
	
\subsubsection{Security and Performance Enhancement Conflict}
%Performance enhancement methods such as PEPs and gateways are affected inversely by adding security features to other layers especially, transport. In other words, adding (security) functionalities to one layer might interfere proper (and mostly performance) operations at the other layers \cite{7005393}. 
Performance enhancement techniques, such as Performance Enhancing Proxies (PEPs) and gateways, play a critical role in ensuring the smooth functioning of IoT networks. However, these techniques often encounter an inverse relationship with the implementation of security measures, particularly at the transport layer \cite{peymanICC16, in-net-res-pool}.

The deployment of security features at one layer can disrupt the operations at other layers or devices on the path towards the destination. An example can be observed in the context of Transport Layer Security (TLS), where encryption at the transport layer can potentially interfere with the efficiency-enhancing operations of PEPs. This issue arises because the encrypted data packets are opaque to the PEPs, preventing them from performing their intended optimization tasks, such as protocol-specific acceleration \cite{7005393}.
	
\subsubsection{Attack Repetition}
%Current security functionalities of IoT protocols and their shortages are commonly known, and each one still needs extensions \cite{yang2017survey}. The overwhelming number of recent attacks such as DDoS in IoT devices show how historical attacks are renewed with even more power.
The current landscape of IoT security functionalities presents a recurring and escalating issue: the reemergence of historical attacks in stronger forms. IoT protocols, despite their advancements, still reveal significant vulnerabilities, necessitating ongoing extensions and enhancements for effective security \cite{yang2017survey}.

The rise of Distributed Denial of Service (DDoS) attacks exemplifies this cycle of attack repetition. As IoT devices increase, they form a vast landscape of potential targets, and unfortunately, unwitting participants in botnet activities. DDoS attacks traditionally overwhelm target systems with an excessive volume of requests, rendering them unable to provide services to legitimate users. In the context of IoT, these attacks have found a renewed form, leveraging the ubiquitous presence and often not enough security of IoT devices to build extensive botnets, amplifying the scale and impact of these attacks.

However, DDoS attacks are just one facet of the problem. The pervasive issue of attack repetition also extends to other forms of security threats, such as eavesdropping, man-in-the-middle attacks, and device spoofing. These age-old attack methodologies are constantly being rehashed, tailored, and optimized to exploit the unique vulnerabilities of IoT environments.
	
\subsubsection{Future Extensions}
%There are also some other issues regarding security of, for example, mobile devices which have not been addressed yet \cite{sicari2015security}. Even securing a protocol/layer in IoT does not mean that it is compliant to other protocols/layers or any future extensions such as mobility, multicast, and QoS which are needed in IoT \cite{Alaba201710}.
The future of IoT promises an expansion of capabilities and functionalities, as well as a broadening of application domains. However, these advancements also imply potential security challenges that need to be thoroughly addressed. While significant strides have been made in securing individual protocols or layers within the IoT architecture, this does not automatically translate into comprehensive security coverage, especially considering interlayer compatibility and future extensions \cite{sicari2015security} \cite{Alaba201710}.

Consider, for instance, the increasing prevalence of mobile devices within the IoT ecosystem. The unique characteristics of these devices—such as their portability, variable connectivity, and resource constraints—pose novel security challenges that are yet to be comprehensively addressed. Moreover, the rise of 5G/6G and beyond technologies, with their emphasis on high-speed, low-latency communication, will only accentuate these concerns.

Future extensions of IoT technologies, such as mobility, multicast, and Quality of Service (QoS), further complicate this security landscape. Mobility brings about dynamic changes in network topology and demands secure, seamless handovers. Multicast communication, on the other hand, requires the secure distribution of data to multiple recipients, a task that becomes increasingly complex as the size and diversity of IoT networks grow. Finally, QoS, which is critical in time-sensitive applications such as autonomous vehicles and telemedicine, demands that security measures do not impede the timely delivery of data.

\subsubsection{Cross-Domain Synergy}
%In cross-domain applications and heterogeneous environments, diversity of protocols and policies and their communication especially from the security point of view is a new challenge.
IoT applications are increasingly spanning across diverse domains, each with their unique requirements, protocols, and security policies \cite{marinakis2018advanced}. This cross-domain nature of IoT applications and the ensuing heterogeneity in the environment poses significant challenges, especially from a security perspective.

Consider the scenario where an IoT network is composed of devices from healthcare, industrial automation, smart home, and automotive sectors. Each of these domains has its unique protocols, data formats, and security requirements. For instance, healthcare devices require strict adherence to privacy regulations, while industrial IoT may prioritize integrity and availability. Smart home devices might need to balance user convenience with security, and automotive IoT demands real-time guarantees to ensure safety. Thus, harmonizing these disparate needs is a nontrivial task.

Cross-domain communication further compounds this issue. When devices from different domains need to interact, they must do so through a common set of protocols and policies. However, designing such universal protocols is challenging due to the need to cater to diverse requirements. Moreover, these protocols should also be robust against potential security threats, as an adversary could exploit the weakest link in the cross-domain communication chain to compromise the entire network.

Beyond the technical issues, there is also the challenge of policy harmonization \cite{bringhenti2021toward}. Different domains often operate under different regulatory frameworks, and reconciling these could be a significant hurdle. For instance, data privacy regulations could vary drastically between healthcare and consumer electronics domains.

\subsection{Challenges of Being both Efficient and Secure}
%From the above discussion, we can infer that there is a trade-off between transport layer efficiency and security in IoT in the aforementioned network stacks. To illustrate the problem via an example, consider IoT applications such as home monitoring and e-health as shown in Fig.~\ref{fig:domains} with the goal of monitoring the health of patient/elderly at home, and in case of emergency, providing a cooperation between these two systems; the level of security (trust/privacy) between patient/elderly monitoring IoT devices, and the center in the hospital, and the trust between these systems and the IoT devices connected to home monitoring is different. In fact, these systems should be connected (e.g. in case of emergency, caregivers should be allowed to enter home). However, the trust inside and between home monitoring and e-health is dependent on many different and temporary situations. In addition, IoT devices do need the support of (CoAP) proxies to be able to talk to the IoT cloud and other systems via an HTTP-like protocol; these proxies might need to break an end-to-end connection or even translate the protocol to another one at the cloud side to efficiently utilize resources at both sides. These all make communications in the IoT very hard to secure if the focus is on performance, or inefficient if security has a high priority. The problem aggravates by the crucial need to a multicast protocol in IoT; although there are proposals for just multicast in IoT such as \cite{huang2016multicast}, securing multicast, however, cannot be currently performed by DTLS \cite{grammatikis2019securing}.
As we examine the complexities of IoT architectures, an inherent trade-off between transport layer efficiency and security in IoT becomes apparent, particularly within the previously discussed network stacks. To elucidate this issue, let's take the case of interconnected IoT applications in the realms of home monitoring and e-health, as depicted in Fig.~\ref{fig:domains}.

These applications are designed with the objective of patient or elderly health monitoring within a residential setting, with the potential for emergency intervention \cite{liu2016smart}. Trust levels and privacy requirements vary significantly across the spectrum of interactions -- between the patient or elderly monitoring IoT devices, the healthcare center, and the home monitoring system. For example, in emergency situations, caregivers should be granted access to the home, yet the trust dynamics within and between the home monitoring and e-health systems are contingent upon various factors, often transient in nature.


\begin{figure}
	\centering
	\includegraphics[width=0.98\linewidth]{figures/Domains.pdf}
	\caption{Different IoT domains: health and home.}
	\label{fig:domains}
\end{figure}

Further complications arise when considering that IoT devices often require the support of CoAP proxies to interface with the IoT cloud and other systems via an HTTP-like protocol. These proxies may need to break an end-to-end connection or even translate the protocol to a different one at the cloud side to optimize resource utilization. This dichotomy -- ensuring efficient communication while maintaining stringent security -- poses significant challenges.

The problem is accentuated by the essential requirement for multicast protocols in IoT. While there have been proposals specifically addressing multicast in IoT, such as the work in \cite{huang2016multicast}, the current landscape lacks the ability to secure multicast through DTLS \cite{grammatikis2019securing}. This gap further emphasizes the challenges in securing communications in the IoT while maintaining efficient performance.

%This discussion underscores the need for a balanced approach that can harmoniously fuse efficiency and security within IoT architectures, without undermining one in favor of the other. 

\section{Recursive InterNetwork Architecture (RINA)}
\label{sec:rina}
\subsection{Introduction}
%RINA \cite{day2008networking} is a novel architecture of networking; it follows a ground-breaking approach to layers in the network protocol stack that we are already familiar with. RINA was first introduced by John Day as a pattern in network architecture in \cite{Day:2008:PNA}. To have a closer look at what RINA is, first we discuss the Internet architecture. The global network which is commonly known as ``The Internet" is the network of networks connecting millions of devices together. Some approaches such as TCP/IP and the OSI model were presented to layer functionalities and provide abstractions. However, the models were deviated during development and improvement, and more importantly, the Internet was not originally designed securely; TCP/IP networks had to adopt many other protocols with redundant functionalities to be able to work. This all has shown to be complicated to manage or secure \cite{small2012}. 
%
%On the contrary, RINA adopts the basic foundation of networking: ``Networking is Inter-Process Communication (IPC) and only IPC'' \cite{day2008networking}. 
%It unifies networking and distributed computing: the network is a distributed application that provides IPC. Moreover,
%it employs a secured layer with basic IPC \textit{mechanisms} (i.e. necessary functionalities), and through a common API, the network administrator is allowed to arrange/stack these secured layers as needed recursively. Each layer is called Distributed IPC Facilities (DIF), and is able to be programmed through \textit{policies} on-the-fly; policies determine how mechanisms could operate.
The Recursive InterNetwork Architecture (RINA), presented by John Day in ``Patterns in Network Architecture: A return to Fundamentals'' \cite{Day:2008:PNA}, is a novel approach in networking architecture trying to avoid architectural problems of TCP/IP and other technologies. Through evaluating it in different use cases (e.g. see (e.g. \cite{peymanICC16,boddapati2012assessing,day2008networking,7510780,leon2016benefits,small2012})), this is shown that RINA can provide significant benefits to networking by removing architectural obstacles\footnote{See \cite{pouzinsociety} for a list of publications on RINA.} \footnote{A comprehensive yet easy-to-read text on RINA can be found in \cite{NGP2019}}. 


As shown in Fig.~\ref{fig:rina}, RINA is based on a single type of layer (represented by dashed boxes in the figure), which is repeated as many times as required by the network designer. The layer is called a Distributed IPC Facility (DIF), which is a distributed application that provides IPC services over a given scope to the distributed applications above (which can be other DIFs or regular applications). These IPC services are defined by the DIF API. All DIFs offer the  same services through their API and have the same components and structure. However, not all the DIFs operate over the same scope and environment nor do they have to provide the same level of services. 

\begin{figure}[!t]
	\centering
	\includegraphics[width=0.98\linewidth]{figures/RINA-crop.pdf}
	\caption{A sample topology represented by DIFs.}
	\label{fig:rina}
\end{figure}

In Fig.~\ref{fig:rina}, there are four devices: Node\,1 and Node\,2 are two end-hosts that are connected via two routers. A1 and A2 are two application processes willing to communicate, and the circles in the DIFs are called IPC Processes (IPCPs). IPCPs have the same structure that can join DIFs. DIF \#2 spans the whole network, allowing packets to be routed and congestion-controlled from A1 to A2 via routers. There are also three DIFs \#1 (could be assumed as the link layer) connecting devices together (doing mostly flow control). However, the structure of the DIFs are the same, and just the scope of the DIFs is different. This solves the problem of inventing new layers (as those presented by MPLS, VLAN, tunneling, etc.) and their deployment; the same layer can be customized via policies and instantiated to operate over different scopes.

In RINA, a Distributed Application is a set of two or more Application Processes (APs) that cooperate to do some function. The set of these APs is called a Distributed Application Facility (DAF). DAFs use DIFs when IPC between the APs in DAF is not possible via shared memory. 

In RINA, invariant parts (\textit{mechanisms}) and variant parts (\textit{policies}) are separated in different components of the architecture. This makes it possible to customize the behavior of a DIF to optimally operate in a certain environment with a set of policies for that environment instead of the traditional ``one size fits all'' approach or having to re-implement mechanisms over and over again.

Fig.~\ref{fig:rina-trans} illustrates the data transfer modules inside IPCPs in a DIF. These modules show in each DIF, there is one (EFCP) sender and one (EFPC) receiver. The IPCP in the middle routes Protocol Data Units (PDUs) received from the left IPCP to the right one via its corresponding port. This scheme repeats at different DIFs, with the difference that it could be longer or as short as a point-to-point link without the routing IPCP. In \cite{peymanICC16}, we evaluated the performance of a recursion of two layers of congestion controllers using a TCP Reno-like controller; we showed that the recursion can improve throughput while decreasing queuing delay compared to an end-to-end TCP Reno and Split TCP (or PEP), where a connection is broken at routers.

The main difference between RINA and other stacks is that RINA recurses the same layer, called DIF numbered from 1 (e.g. 1-DIF, 2-DIF) as the lowest one, and (N)-DIF as the current one we are focusing on. At the lowest layer, \textit{shim} DIF operates over the physical layer, but it has the deployment possibility of operating on other protocols such as UDP.

\begin{figure}[!t]
	\centering
	\includegraphics[width=0.98\linewidth]{figures/Transport.pdf}
	\caption{Data transfer modules of three IPCPs inside one DIF.}
	\label{fig:rina-trans}
\end{figure}


%\subsection{IPCP's Data Transfer Mechanisms}
%A node joins a DIF through an IPC Process (IPCP), which is an instance of the same code handling IPC. IPCPs have the same structure consisting of the following mechanisms that operate at different timescales (ordered from faster to slower):
%\begin{enumerate}
%	\item Data Transfer: handles packet transmission including:
%	\begin{itemize}
	%		\item Delimiting: encodes Service Data Units (SDUs) that arrive from the upper DIF within PDUs.
	%		\item Error and Flow-Control Protocol (EFCP): handles data transmission using its two sub-protocols: Data Transfer Protocol (DTP) and Data Transfer Control Protocol (DTCP).
	%		\item Relaying and Multiplexing Task (RMT): performs routing of PDUs to output ports of the DIF or upwards.
	%		\item SDU Protection: performs encryption, compression, error-code and TTL.
	%	\end{itemize}
%	\item Data Transfer Control: manages error, flow, and retransmission control.
%%	\item Layer Management: includes
%%	\begin{itemize}
	%%		\item Routing,
	%%		\item Common Distributed Application Protocol (CDAP): operates on configuration objects, and layer management,
	%%		\item Resource and flow allocation,
	%%		\item Locating applications,
	%%		\item Security management, access control,
	%%		\item Enrollment, and authentication.
	%%	\end{itemize}
%\end{enumerate}

\subsection{IPCP's Mechanisms}

A node joins a DIF through an IPC Process (IPCP), which is an instance of the same code handling IPC. IPCPs have the same structure consisting of the following mechanisms that operate at different timescales (ordered from faster to slower):

\subsubsection{Data Transfer}
This is the most fundamental mechanism of the IPCP, and it is responsible for the actual movement of data. It involves a series of functions:
\begin{itemize}
	\item \textit{Delimiting}: This function encodes Service Data Units (SDUs) that arrive from the upper DIF within Protocol Data Units (PDUs).
	\item \textit{Error and Flow-Control Protocol (EFCP)}: This protocol ensures reliable data transmission using its two sub-protocols: Data Transfer Protocol (DTP) and Data Transfer Control Protocol (DTCP). DTP primarily concerns with the transmission and segmentation of data, while DTCP provides additional control functions, such as congestion control and error checking.
	\item \textit{Relaying and Multiplexing Task (RMT)}: This task performs routing of PDUs to output ports of the DIF or upwards. It utilizes the routing information provided by the layer management to determine the optimal path for data transmission.
	\item \textit{SDU Protection}: This mechanism performs encryption, compression, error-code calculation, and Time-To-Live (TTL) setting.
\end{itemize}

\subsubsection{Data Transfer Control} 
This mechanism supervises the data transmission process to ensure it runs smoothly and effectively. It includes several sub-components:
\begin{itemize}
	\item \textit{Error Control}: This component keeps track of any errors that may occur during data transmission. It uses a variety of error detection and correction techniques to maintain data integrity throughout the transmission process.
	\item \textit{Flow Control}: This component manages the rate of data transmission ensuring that the receiver is not overwhelmed with data and can process incoming data effectively.
	\item \textit{Retransmission Control}: This component monitors for lost or corrupted data packets. In the event of packet loss or corruption, it initiates a retransmission request, ensuring reliable data delivery.
	\item \textit{Congestion Control}: This component monitors network congestion and implements measures to alleviate it. This can include reducing the data transmission rate.
\end{itemize}

\subsubsection{Layer Management}
The layer management mechanism includes a variety of functions:
%\begin{itemize}
%	\item \textit{Routing}: This function determines the most effective route for data packets across the network.
%	\item \textit{Common Distributed Application Protocol (CDAP)}: This protocol operates on configuration objects and manages layer functions.
%	\item \textit{Resource and Flow Allocation}: This function allocates the necessary network resources and manages data flow.
%	\item \textit{Locating Applications}: This function allows the network to identify the location of specific applications.
%	\item \textit{Security Managementl}: These functions are responsible for ensuring that only authorized individuals can access the network and its resources.
%	\item \textit{Enrollment and Authentication}: These processes verify the identity of nodes or users, ensuring that only authorized entities are able to join the network and communicate within it.
%\end{itemize}

\begin{itemize}
	\item \textit{Resource Allocation:} The Resource Allocator (RA) manages resource allocation and monitors the resources in the DIF by sharing information with other DIF IPC Processes and the performance of supporting DIFs.
	\item \textit{Routing:} Routing performs the analysis of the information maintained by the RIB to provide connectivity input to the creation of a forwarding function. The choice of routing algorithms in a particular DIF is a matter of policy.
	\item \textit{Security Coordination:} Security coordination is responsible for implementing a consistent security profile for the IPC Process, coordinating all the security-related functions (authentication, access control, confidentiality, integrity) and also executing some of them (auditing, credential management).
	\item \textit{Namespace Management:} The Name Space Manager (NSM) embedded in the DIF is responsible for mapping application names to IPC Process addresses. The NSM maintains a mapping between external application names and IPC Process addresses where there is the potential for a binding within the same processing system.
	\item \textit{Flow Allocation:} The Flow Allocator is responsible for creating and managing an instance of IPC, i.e., a flow. The Flow Allocator-Instance (FAI) determines what policies will be utilized to provide the characteristics requested in the Allocate.
	\item \textit{Enrollment:} The Enrollment process involves a new member joining the DIF. This process includes address assignment, information exchange about operational parameters, and updating the RIB with the latest information on routing, directory, resource allocation, etc. The new member then becomes a full member of the DIF.
\end{itemize}


%\begin{figure} 
%	\centering
%	\includegraphics[width=0.47\textwidth]{figures/RINA-topology.pdf}
%	\caption{A sample RINA topology with two end-nodes and two routers. Every IPCP has the same internal structure.}
%	\label{fig:RINA-topology}
%\end{figure}
%
%Referring to Fig.~\ref{fig:RINA-topology}, the dashed arrows show the path of data/message exchange between the two applications in nodes 1 and 2. Every IPCP decides how to process a received PDU from the upper/lower layer; it can pass it to the lower layer, send it to the other IPCP if the DIF is on the physical medium (i.e. it is a 1-DIF), routes it to another IPCP in a lower DIF if it knows where the destination is (e.g. the IPCPs in 2-DIF in the routers), or pass it upwards to the destination application (at node 2).
%Therefore, there is a single type of layer with programmable functions, that repeats as many times as needed by network designers. It means that
%all layers provide the same mechanisms: instances or communication (flows) between two or more application instances, with certain characteristics (delay, loss, in-order-delivery, etc). However, the mechanisms are programmable/customizable through policies. In general, there are only 3 types of nodes in RINA: hosts, interior\footnote{Interior routers are not shown in the figure. See \cite{7510780} for a complete topology.} and border routers, and there is no need for middleboxes such as firewalls, NATs, etc. because policies can customize the internal behavior of each IPCP (or DIF), which consequently empowers nodes with any required functionality.

%Through a number of research work\footnote{A complete list of publications on RINA can be found in http://www.pouzinsociety.org/research/publications} and international projects\footnote{See http://www.pouzinsociety.org/research/projects}, it has been shown that RINA can effectively improve the network performance in terms of throughput and delay \cite{peymanICC16}, quality of service \cite{gaixas2016assuring}, reducing the size of routing tables \cite{leon2016benefits, hrizi2017hierarchical, hrizi2015sfr}, multi-homing and mobility \cite{ishakian2012supporting}. These benefits are all very appealing for IoT networks \cite{trouva2010internet}. In the context of sensor networks, a framework that provides common features collected from different types of sensor networks to describe the interrelations 
%among the entities in a sensor network and possibly between different sensor networks is required \cite{ISO29182}. We believe that RINA can provide a promising approach toward this objective as well \cite{trouva2010internet}.

\subsection{Error and Flow-Control Protocol (EFCP)}
EFCP is the only data transfer protocol in RINA, which is based on Richard Watson's fundamental results on synchronization for reliable data transfer \cite{watson1981timer}. Its main functions are sequencing, flow control, and retransmission control. EFCP provides an inter-process communication service to upper processes that might be an (N+1)-IPCP or and application process. The upper process is connected to (can write to/read from) the EFCP in the (N)-DIF via the (N)-port-id.

EFCP has two logical components operating at different time scales: 1) Data Transfer Protocol (DTP) that is the mandatory part of EFCP and includes tightly bound mechanisms, and 2) Data Transfer Control Protocol (DTCP), which include loosely bound mechanisms. DTP is instantiated every time a flow is created. It is roughly similar to the UDP protocol. However, depending on the QoS requirements, DTCP might be also instantiated to provide retransmission and flow control.

Watson's result imply that the necessary and sufficient condition for synchronization to have a reliable data transfer is to set an upper bound on only these times: Maximum Packet Lifetime (MPL), maximum time to wait before sending an Ack, A, and the maximum time to exhaust retries, R \cite{watson1981timer}. This decouples port allocation and synchronization.

EFCP can be customized via several policies. A simple example is that the behavior of different TCP flavors can be imitated by writing different policies. The mechanisms operating on PDUs and policies are the same.

As shown in Fig.~\ref{fig:rina-efcp} , when a PDU is written to an (N)-port-id, the delimiting module, associated to the port, processes the PDU by possible fragmentation or concatenation. The result is passed to the EFCP instance associated to that port. As a result, generated PDUs are passed to the RMT for multiplexing. The RMT sends PDUs to one or more (N-1) ports. When the RMT reads a PDU from an (N-1)-port, if the PDU's destination address is not in another node, it passes the PDU to the relevant EFCP instance for further processing. EFCP instances of (N)-DIF are managed by the module Flow Allocator (FA) in that DIF. FA can create a new EFCP instance, replace an old instance by a new one, or delete the instances.

\begin{figure}
	\centering
	\includegraphics[width=0.98\linewidth]{figures/EFCP.pdf}
	\caption{A sample of data transfer modules.}
	\label{fig:rina-efcp}
\end{figure}


\section{RINA's Transport Security Features} \label{sec:rina-trans}
How RINA connects IPCPs and how their EFCP is connected to each other in different DIFs can provide other benefits than just performance. A complete evaluation is found in \cite{ramezanifarkhani2018securing,small2012}. Here, we summarize the security features related to EFCP:

\subsection{Secure DIFs}
``DIF is a securable container'' \cite{boddapati2012assessing}. This is the main feature that RINA secures layers instead of protocols. This means that all the packets leaving an (N)-DIF are protected via the SDU Protection module. Hence, IPCPs in (N-1)-DIF cannot inspect the arriving packet, and this can continue downwards in lower DIFs, which also depends on the policy of those DIFs. However, every IPCP in the same DIF can inspect the packet, and in case of performing enhancements or protocol conversion in CoAP proxies, packets are not obscured. This also implies that in RINA, IPCPs should be enrolled first before joining a DIF, and their access rights are validated. This also implies that an outsider cannot attack the DIF. In case an attacker can compromise the enrollment or if the DIF communication is not encrypted, RINA's typical field lengths in packets are still long enough to make attacks harder to succeed, e.g. $2^{48}$ possibilities to guess the connection information in RINA compared with $2^{29}$ possibilities in TCP during data transfer \cite{boddapati2012assessing}.

\subsection{Hidden Addresses}
%RINA has a special addressing rule: IPCPs in each DIF have their own addresses, meaning that addresses are hidden from other DIFs. This can mitigate the problem that IP addresses are public in the Internet without any authentication. 
RINA introduces a unique addressing scheme where IPCPs in each DIF have their own addresses. Unlike the traditional IP addressing system, where IP addresses are globally visible and can be targeted by malicious actors, in RINA, the addresses are confined to their respective DIFs and are not publicly visible.

This ``hidden address'' feature effectively mitigates a major security concern in traditional IP-based networks, where IP addresses are publicly accessible without any form of authentication. Publicly visible IP addresses expose the network to a variety of threats, such as Denial of Service (DoS) attacks, IP spoofing, and other forms of unauthorized access or data interception.

In an IoT context, where networks are typically composed of a large number of interconnected devices with varying levels of security capabilities, this feature of hidden addresses can offer substantial security benefits:

\begin{itemize}
	\item \textit{Enhanced Privacy}: By keeping the addresses hidden within each DIF, RINA effectively reduces the attack surface, making it more difficult for malicious actors to target specific devices within the network.
	\item \textit{Scalability}: The RINA recursive nature allows for highly scalable networks without compromising security, essential for large-scale IoT deployments.
	%	\item \textit{Authentication and Access Control}: In a RINA network, each DIF operates its own namespace and handles authentication independently. This feature allows for more granular and reliable control over who can access what in the network, which is crucial in an IoT environment.
	%	\item Network Segmentation: The compartmentalization of addresses within DIFs naturally supports network segmentation, which is a best practice in network security. This segmentation can prevent the lateral movement of threats within the network, containing potential security incidents.
\end{itemize}

\subsection{Synchronization-Independent Port Allocation}
%In TCP, synchronization and port allocation are coupled, which makes it vulnerable. However, the decoupling of these two in RINA reduces the chance of intercepting a connection, and as a result, attacks are harder to mount. %In RINA, there is no well-known ports to listen to; applications are requested for service through their application name.
In TCP, synchronization and port allocation processes are coupled. A potential vulnerability arises from this coupling because it becomes easier for an attacker to predict or ascertain a connection's state and launch attacks, such as SYN flood attacks, which are a type of Denial of Service (DoS) attack.

However, RINA introduces a novel design where synchronization and port allocation are decoupled, which significantly enhances the security posture of IoT networks. This decoupling makes it more difficult for attackers to predict or understand the state of a connection, hence reducing the chances of intercepting a connection or launching successful attacks.

%Additionally, unlike in traditional IP networks where services are associated with well-known ports (for instance, port 80 for HTTP or port 443 for HTTPS), in RINA, there are no such well-known ports to listen to. This absence of well-known ports further decreases the attack surface, as attackers cannot target these ports with port-specific attacks or scan them for vulnerabilities.

%Instead, in RINA, applications request services through their application names. This approach provides several security advantages:
%
%\begin{itemize}
%	\item \textit{Enhanced Privacy and Security}: As there are no well-known ports to target, it becomes significantly harder for an attacker to identify and exploit potential vulnerabilities or intercept communication.
%	\item \textit{Application-Specific Security Policies}: Services can be secured based on their application names, allowing for more granular and application-specific security policies. This approach can provide robust protection tailored to each application's unique security requirements.
%	\item \textit{Name-Based Routing}: Since routing is based on names rather than addresses, it is more difficult for an attacker to perform IP spoofing or Man-in-the-Middle (MitM) attacks.
%	\item \textit{Stronger Authentication}: By using application names for service requests, the network can implement stronger authentication mechanisms, thereby ensuring that only authorized applications can establish connections.
%\end{itemize}

\subsection{Port-Independent Communication}
%In RINA, there is no well-known ports to listen to; applications are requested for service through their application name. RINA decouples port allocation and access control from data synchronization and transfer which makes attacks harder to mount \cite{small2012}.
In most traditional network architectures, including the Internet Protocol (IP) suite, services are associated with well-known ports. For example, web servers typically listen on port 80 for HTTP and port 443 for HTTPS. These well-known ports can be targeted by attackers, who scan them for vulnerabilities or attempt to overwhelm them with traffic in Denial-of-Service (DoS) attacks.

However, RINA adopts a fundamentally different approach. Instead of relying on well-known ports, applications request services through their application names. This means there are no standard ports for attackers to listen to or target, significantly reducing the attack surface \cite{small2012}.

\subsection{Soft-State Connection Management}
%As stated before, RINA adopts Watson's method. This means that there is no explicit control messages for connection establishment/close. There are only timers, and a receiver deletes the state after 2$*$MPL (Maximum Packet Lifetime). This reduces the chance of connection misuse \cite{boddapati2012assessing}.
%%	\item \label{rina:resist-TL-att} RINA uses a wider range of field values. This means that without the support of cryptography, RINA is more resistant to transport-layer attacks faced by TCP/IP \cite{boddapati2012assessing}.
In RINA, the management of connections is handled differently from traditional network architectures. RINA adopts Watson's method for managing connections in a soft-state manner. This means that instead of using explicit control messages to establish or close a connection, RINA relies on timers. After a period of time equal to twice the Maximum Packet Lifetime (2$*$MPL), the receiver deletes the connection state. This unique feature of RINA offers several significant security benefits for IoT networks:

\begin{itemize}
	\item \textit{Reduction of Connection Misuse}: With the absence of explicit control messages for connection establishment and termination, the risk of connection misuse is significantly reduced. Misuse can occur when malicious actors manipulate control messages to illegitimately establish, hijack, or disrupt connections. By eliminating these messages, RINA reduces the opportunities for such attacks \cite{boddapati2012assessing}.
	\item \textit{Mitigation of Denial of Service (DoS) Attacks}: Many DoS attacks, such as SYN flood attacks, exploit the control messages used to establish connections in traditional network architectures. RINA's soft-state connection management approach effectively mitigates this type of attack by eliminating the targeted control messages.
	\item \textit{Improved Resource Management}: By automatically deleting connection states after 2$*$MPL, RINA ensures efficient use of resources. This feature is particularly important in IoT networks, where devices often have limited computational resources. Efficient resource management also reduces the risk of resource exhaustion attacks.
	\item \textit{Enhanced Privacy}: By not maintaining long-lived connection states, RINA also improves privacy. In traditional networks, persistent connection states could potentially be used to track and monitor user activities. In RINA, the transient nature of connection states makes such tracking more difficult.
\end{itemize}

\subsection{Connection Management Independent Authentication}
%Since IPCPs are first enrolled in a DIF, authentication is not a part of connection management in RINA.
%%	\item RINA decouples port allocation and access control from data synchronization and transfer which makes attacks harder to mount.
RINA implements a unique architectural design where the authentication process is separate from connection management. Instead of performing authentication during connection establishment as in traditional network architectures, IPCPs in RINA are authenticated when they first enroll in a DIF.

This decoupling of authentication from connection management brings about several significant security benefits for IoT networks:

\begin{itemize}
	\item \textit{Enhanced Security}: By performing authentication at the enrollment stage, RINA ensures that only authenticated IPCPs can participate in the network. This measure significantly reduces the risk of unauthorized access or impersonation attacks.
	\item \textit{Minimized Attack Surface}: Traditional network architectures often couple authentication with connection management, which can expose them to a variety of attacks such as man-in-the-middle (MitM) attacks. By decoupling these processes, RINA minimizes the attack surface and makes it harder for attackers to exploit the connection management process.
	\item \textit{Robust Access Control}: With the separation of authentication from connection management, RINA can implement more robust access control mechanisms. This allows for more granular control over who can access what in the network, which is critical in IoT environments where devices with varying security capabilities are interconnected.
	\item \textit{Efficient Resource Management}: By handling authentication during enrollment, RINA can also prevent resource exhaustion attacks aimed at overwhelming the authentication process during connection establishment.
\end{itemize}

\subsection{Variable Address Space}
%In addition to the addresses that are invisible to other DIFs, the size of addresses can vary in DIFs, which makes it harder for attackers. The address size, however, depends on the number of nodes in that DIF to save space in the packet header.
In traditional network architectures, addresses are usually fixed in size. For instance, IPv4 addresses are always 32 bits, and IPv6 addresses are 128 bits. These fixed-sized addresses can be predictable and therefore potentially exploitable by attackers.

RINA, on the other hand, introduces the concept of variable address space. The size of addresses in a DIF can vary, and it depends on the number of nodes in that DIF. This approach not only saves space in the packet header but also introduces a significant layer of unpredictability, making it harder for attackers to predict or manipulate addresses.

Here are some of the security benefits provided by the variable address space feature in RINA:

\begin{itemize}
	\item \textit{Enhanced Unpredictability}: The variability in address size makes it difficult for attackers to predict the address space, thereby making address-based attacks such as IP spoofing and reconnaissance more difficult.
	\item \textit{Increased Difficulty for Eavesdroppers}: The variability in address size also makes it harder for eavesdroppers to understand the network's structure or to track specific devices, thus enhancing the privacy of the network.
	\item \textit{Efficient Use of Address Space}: By adjusting the address size according to the number of nodes in a DIF, RINA ensures efficient use of the address space. This is particularly beneficial in IoT networks, where the number of devices can vary significantly.
	\item \textit{Improved Scalability}: The ability to vary address size provides RINA with superior scalability, ensuring that the network can effectively support both small-scale and large-scale IoT deployments.
\end{itemize}

\section{Other Performance/Security Features of RINA} 
\label{sec:rina-other}
\subsection{RINA's DAFs}
In RINA, the application-layer component is encapsulated in what is known as the Distributed Application Facility (DAF). The DAF is responsible for providing services to applications and managing application-level communication. From a security standpoint, the DAF in RINA embodies the principle of end-to-end security. Unlike other networking architectures where security is often implemented as an afterthought, RINA incorporates security as an integral part of the DAF design. This means that applications using a DAF can rely on a built-in security model that protects their communication from eavesdropping, tampering, or message forgery.

When a DAF encrypts all communication, it indeed provides a robust layer of security for the application data, essentially rendering the information within the DIFs secure. This approach aligns with the principle of data protection at rest and in transit, one of the cornerstones of information security \cite{kunchok2018lightweight}. However, it's essential to understand that within a RINA network, each DIF still plays a crucial role in maintaining the overall system's security. For instance, the top DIF might employ Payment Card Industry (PCI) encryption \cite{liu2010survey} to prevent traffic analysis, a form of network surveillance that threatens privacy. This layer-specific encryption is a proactive measure to counteract any potential security breaches.

It can be argued that that link-layer encryption becomes redundant in this context, as it introduces an overhead of encrypting and decrypting at every hop, especially when the addresses used at this layer are not usually of interest to potential attackers.

\subsection{RINA's CDAP}
RINA's architectural design emphasizes the importance of a unified application protocol and the flexibility of object models. One of the significant advantages of RINA is its Common Distributed Application Protocol (CDAP). It provides the necessary framework for creating a wide range of distributed applications, with the DIF being a notable instance of such an application.

Unlike the traditional approach of having different application protocols, which often results in protocol inconsistencies, RINA's CDAP establishes a common language for all applications, simplifying communication across varied applications.

The CDAP's generic nature provides a platform for diverse applications to exchange information using a unified protocol. This consistency removes the need for multiple application-specific protocols, such as the Constrained Application Protocol (CoAP), thereby reducing the complexity and overhead associated with maintaining numerous protocols. The elimination of CoAPs altogether aids in streamlining the intercommunication between IoT devices and servers, enhancing both the efficiency and reliability of data exchange.

\subsection{Multi-Layer Security}
RINA embraces the ``Divide and Conquer'' principle by using DIFs and recursion. Instead of trying to secure a broad scope, such as an entire IoT network at once (which could be as vast as the Internet), security is enforced within smaller, more manageable scopes defined by DIFs.

This approach yields several significant advantages:

\begin{itemize}
	\item \textit{Containment of Attacks}: By dividing the network into smaller scopes (DIFs), the impact of a compromise can be contained within a single DIF. This means that even if some DIFs are compromised, the integrity of the whole network remains intact. This is a critical benefit in IoT networks where a vast number of diverse devices, each with their own potential vulnerabilities, are interconnected \cite{day2008networking}.
	\item \textit{Enhanced Scalability}: The recursive nature of DIFs means that the same security mechanisms can be applied at different scales, from small local networks to large-scale global networks. This scalability is crucial for IoT networks, which can range from a few devices in a home network to millions of devices in a city-wide infrastructure.
	\item \textit{Improved Manageability}: Securing smaller scopes (DIFs) is more manageable than securing a wide network as a whole. This allows for more targeted security measures and makes it easier to monitor and respond to security incidents.
	\item \textit{Tailored Security Policies}: Each DIF can have its own security policies tailored to the specific needs and characteristics of the devices within it. This allows for a more fine-grained and effective approach to security compared to one-size-fits-all policies.
\end{itemize}

\subsection{Communication via a Common DIF}
%On the contrary to the Internet, two applications are only able to communicate if they have a DIF in common. Otherwise, they should join or create a common DIF. 
In RINA, two applications can only communicate if they share a DIF. If they do not have a DIF in common, they must either join an existing common DIF or create a new one. This is a departure from traditional Internet protocols, where any two nodes with Internet connectivity can attempt to communicate.

This feature offers several significant advantages for IoT network security:

\begin{itemize}
	\item \textit{Restriction of Unwanted Communication}: By requiring a common DIF for communication, RINA effectively restricts the ability of arbitrary nodes to communicate with each other. This can significantly reduce the risk of unwanted communication, such as those originating from malicious actors or compromised devices.
	\item \textit{Enhanced Access Control}: The requirement of a common DIF also provides a mechanism for fine-grained access control. Only devices that are part of a given DIF can communicate, enabling network administrators to effectively control which devices can interact with each other.
	\item \textit{Reduced Attack Surface}: By limiting communication to devices within a common DIF, RINA effectively reduces the attack surface of an IoT network. Attackers cannot directly reach devices outside of their DIF, which can help protect vulnerable devices from attacks.
	\item \textit{Improved Privacy}: Since a device can only communicate with other devices within its DIF, the visibility of device communication is limited. This can help enhance the privacy of device interactions within an IoT network.
\end{itemize}

\subsection{Authentication}
%Every IPCP should be authenticated first before joining a DIF. This is performed before connection management through the enrollment process, and enrollment does include access control. This means that attackers have to join a DIF to be able to address IPCPs in that DIF which requires authentication first.
In RINA, every IPCP must be authenticated before joining a DIF. This authentication is performed during the enrollment process, prior to connection management. Notably, the enrollment process also includes access control measures.

These security mechanisms have several significant implications for IoT network security:

\begin{itemize}
	\item \textit{Enhanced Access Control}: Requiring IPCP authentication before joining a DIF ensures that only trusted entities can become part of the network. This measure effectively prevents unauthorized entities from joining the network, thereby reducing the risk of insider threats and unauthorized access.
	\item \textit{Mitigation of Address Spoofing Attacks}: Since attackers must join a DIF (which requires authentication) to address IPCPs within that DIF, RINA's architecture inherently mitigates the risk of address spoofing attacks. 
	\item \textit{Improved Network Integrity}: By enforcing authentication and access control during the enrollment process, RINA enhances the overall integrity of the network.
	\item \textit{Prevention of Unauthorized Data Access}: With strict authentication and access control, unauthorized entities are prevented from accessing or manipulating sensitive data within the network. This is crucial in IoT networks, where vast amounts of potentially sensitive data are regularly transmitted.
\end{itemize}

\subsection{Built-in Firewall}
%Every router will naturally play as a firewall in RINA. (Security) modules in IPCPs can provide firewall functionalities. %RINA only has three types of nodes: end-nodes, border routers, and interior routers.
In RINA, every router inherently serves as a firewall. This is due to the security modules present within each IPCP. These security modules can provide firewall functionalities, thus enhancing the security of data transiting through the network.

This feature offers several significant advantages for IoT network security:

\begin{itemize}
	\item \textit{Enhanced Network Security}: The inherent firewall functionality in each router allows for enhanced protection against various network attacks. This includes protection against unauthorized access, data breaches, and various forms of cyber threats.
	\item \textit{Simplified Network Architecture}: RINA simplifies the network architecture by having only three types of nodes: \textit{end-nodes}, \textit{border routers}, and \textit{interior routers}. This streamlined architecture simplifies the implementation of security policies and reduces the potential attack surface.
	\item \textit{Distributed Security}: Rather than relying on a centralized security solution, RINA's approach distributes security functionalities across all routers in the network. This can help to prevent single points of failure and distribute the load of handling security functions.	
	\item \textit{Granular Security Control}: The presence of security modules within each IPCP enables granular control over security policies. This means that different routers can implement different levels of firewall protection based on the specific needs and risk levels of their respective network segments.
\end{itemize}

\subsection{Programmable DIFs}
%Any new functionality, which might address some security, privacy, or performance issue, can be simply developed as a \textit{policy} and plugged into existing mechanisms. This reduces functional redundancies in protocols and the risk of causing new vulnerabilities by reducing required efforts.
RINA enables DIFs to be programmable. This means that any new functionality, including those addressing security, privacy, or performance issues, can be developed as a policy and plugged into the existing mechanisms.

The implications of programmable DIFs for IoT network security are substantial:

\begin{itemize}
	\item \textit{Flexibility in Security Policy Implementation}: By allowing new functionalities to be developed as policies, RINA offers a high degree of flexibility in implementing and updating security measures. This can be especially beneficial in the rapidly evolving landscape of IoT security, where new threats and vulnerabilities often emerge \cite{s20133622}.
	\item \textit{Reduction of Functional Redundancies}: The ability to plug in policies into existing mechanisms can help reduce functional redundancies in protocols. This not only streamlines the network's operation but also eliminates potential security loopholes that might arise from redundant functionalities \cite{en14102818}.
	\item \textit{Mitigation of New Vulnerabilities}: By reducing the effort required to implement new functionalities, RINA lowers the risk of introducing new vulnerabilities during the implementation process. This is crucial because even minor errors in implementing security measures can lead to significant vulnerabilities.	
	\item \textit{Adaptable Security Measures}: The programmable nature of DIFs in RINA allows the security measures to be readily adaptable to the specific needs and threat landscape of an IoT network. This means that the security measures can be rapidly adjusted in response to evolving threats and vulnerabilities.
\end{itemize}

\subsection{Access Control}
%Authorization in RINA is performed by the Access Control module in IPCP, and uses CDAP as the signaling protocol. This mechanism determines if a requesting entity is allowed to access a given resource.  
In RINA, access control is enforced by the Access Control module within the IPCP, using CDAP as the signaling protocol. This mechanism determines whether a requesting entity is permitted to access a given resource.

The implications of RINA's access control mechanism for IoT network security are considerable:

\begin{itemize}
	\item \textit{Authorization}: By using the Access Control module, RINA ensures that every entity requesting access to a resource must be authorized. This adds an extra layer of protection, effectively reducing the risk of unauthorized access to network resources.
	\item \textit{Fine-Grained Access Control}: Given that the access control is enforced at the IPCP level, this allows for fine-grained access control. Different resources can have different access control policies, thus providing a high level of granularity in controlling access to network resources.
	\item \textit{Protection Against Insider Threats}: As the access control is enforced for each resource access request, this can effectively mitigate the risk of insider threats. Even if an entity is authenticated, it must still be authorized to access a resource, preventing misuse of access privileges.
	\item \textit{Robust Signaling Protocol}: The use of CDAP as the signaling protocol for access control further enhances the security of the mechanism. CDAP is a robust protocol that has built-in measures to ensure the integrity and authenticity of signaling messages, thus improving the overall security of the access control mechanism.
\end{itemize}

\subsection{Insiders Resistance}
%RINA uses a wider range of control field values (e.g., connection/QoS id). Given that an attacker can somehow compromise authentication or without the support of cryptography, RINA's typical field lengths in packets are still long enough to make attacks harder to succeed. It has been analyzed how hard and effective it can be, e.g., $2^{48}$ possibilities to guess
%the connection information in RINA compared with $2^{29}$ possibilities in TCP during data transfer \cite{boddapati2012assessing}.
%%\red{a possibble attack: We analyze how hard it is to compromise RINA given typical field lengths in packets in \cite{boddapati2012assessing}} \peyman{Toktam: please check the above item.}
RINA adopts a wider range of control field values, for instance, in connection and Quality of Service (QoS) identifiers. This strategy is specifically designed to resist insider threats, those threats posed by entities that have already gained access to the network or have somehow bypassed the authentication process, even in scenarios where cryptographic support might be absent.

The implications of RINA's insiders resistance for IoT network security are as follows:

\begin{itemize}
	\item \textit{Enhanced Security Through Larger Field Lengths}: The typical field lengths in RINA packets are longer than those in traditional IP-based protocols like TCP. This increases the complexity and computational effort required for an attacker to guess the correct values \cite{small2012}.
	\item \textit{Increased Difficulty in Guessing Connection Information}: As stated in \cite{boddapati2012assessing}, the field lengths in RINA offer $2^{48}$ possibilities for connection information. This is substantially larger than the $2^{29}$ possibilities in TCP during data transfer, making it exponentially harder for an attacker to guess the connection information.
%	\item \textit{Improved Resistance to Insider Attacks}: By using larger field lengths, RINA improves the network's resistance to insider attacks. Even if an attacker can compromise the authentication process, the increased complexity and computational effort required to guess the correct field values act as a formidable barrier to successful attacks.
%	\item \textit{Strengthened Network Security}: Overall, the wider range of control field values in RINA strengthens the security of the IoT network by providing an additional layer of protection against attacks, particularly those from insider threats.
\end{itemize}

\subsection{QoS}
%Every connection in RINA is established after the source represents its QoS requirements which include maximum requested bandwidth \cite{gaixas2016assuring}. Deviating from those, e.g. in DoS attacks by congesting the network, can result in dropping its packets at the first routing node, which is some form of DoS prevention. %Also, it has been shown that RINA can effectively improve quality of service .
In RINA, each connection is established only after the source presents its QoS requirements, which typically include parameters such as the maximum requested bandwidth as outlined in \cite{gaixas2016assuring}. This capability of RINA has important implications for IoT network security:

\begin{itemize}
	\item \textit{DoS Prevention}: A major security benefit of the RINA's QoS mechanism is its inherent resistance to Denial of Service (DoS) attacks. If an entity attempts to deviate from its specified QoS requirements, such as by generating excessive network traffic in an attempt to congest the network, its packets can be dropped at the first routing node. This acts as a form of DoS prevention, as it allows the network to maintain its service availability even in the face of potential attacks.
	\item \textit{Traffic Regulation}: By enforcing QoS requirements at the connection level, RINA helps regulate network traffic. This means that each connection can only generate a certain amount of traffic, as specified by its QoS requirements. This helps prevent network congestion and maintains the overall performance and reliability of the network.
%	\item \textit{Network Resource Management}: QoS requirements also play an important role in managing network resources. By clearly defining the network resources (like bandwidth) that each connection can use, RINA ensures that network resources are distributed fairly and efficiently among all connections.	
	\item \textit{Monitoring and Detection}: The enforcement of QoS requirements can also serve as a basis for monitoring and detection mechanisms \cite{protogerou2021graph}. Any deviation from the specified QoS requirements can be flagged as potential malicious activity, enabling the early detection and mitigation of security threats.
\end{itemize}

\subsection{Resiliency}
%In each DIF, (multi-path) routing is performed independently and transparently to the other DIFs. This means that each DIF can provide resiliency services as well to the upper DIFs. In addition, this properties provides ``transport over heterogeneous networks" \cite{Trouva:2011:TOH}.
In RINA, each DIF independently and transparently performs (multi-path) routing to the other DIFs. This design brings about several significant implications for IoT network security:

\begin{itemize}
	\item \textit{DIF-Level Resiliency}: Each DIF is capable of providing resiliency services to the upper DIFs. This means that even if a particular path or network segment becomes unavailable, the communication can still proceed via alternative paths, enhancing the overall reliability and availability of the network.
	\item \textit{Heterogeneous Network Transport}: As noted in \cite{Trouva:2011:TOH}, this design also enables ``transport over heterogeneous networks''. This property allows IoT networks to operate over and across different types of networks, enhancing their interoperability and adaptability to different environments and requirements.
	\item \textit{Mitigation of Single Points of Failure}: The independent routing mechanism of each DIF effectively decentralizes the network's routing process. This decentralization reduces the risk of single points of failure and makes the network more robust against attacks that target specific network components or paths.
	\item \textit{Load Balancing}: Multi-path routing can distribute network traffic across multiple paths, helping to balance the load and prevent network congestion \cite{marek2019high}. This load balancing can further enhance network performance and reliability, making it more difficult for attackers to disrupt the network through congestion-based attacks.
	\item \textit{Rapid Recovery from Failures}: The resiliency feature also enables rapid recovery from network failures. If a particular path or DIF fails, the independent routing mechanism can quickly reroute the traffic through other operational paths or DIFs, minimizing the impact of the failure on the network's operation and service availability \cite{neelam2021applicability}.
\end{itemize}

\subsection{Performance Improvements}
%In addition to the above security features, RINA has some other important features which are all very appealing for IoT networks \cite{trouva2010internet}. For example, through a number of research work\footnote{A complete list of publications on RINA can be found in http://www.pouzinsociety.org/research/publications} and international projects\footnote{See http://www.pouzinsociety.org/research/projects}, it has been shown that RINA can effectively improve the network performance in terms of throughput and delay \cite{peymanICC16}.
Beyond the security benefits provided by RINA, the architecture also possesses some crucial features that markedly improve network performance, which are particularly attractive for IoT networks \cite{trouva2010internet}. Various research studies\footnote{A comprehensive list of publications on RINA can be found at \underline{http://www.pouzinsociety.org/research/publications}} and international projects\footnote{For more information, visit \underline{http://www.pouzinsociety.org/research/projects}} have demonstrated that RINA can significantly enhance network performance in terms of throughput and delay \cite{peymanICC16}. These performance improvements indirectly contribute to IoT security in several ways as discussed in the above subsections.

%\begin{itemize}
%	%	\item Enhanced Detection and Response to Attacks: Improved network performance can facilitate the rapid detection and response to potential security threats. A network with superior throughput and reduced delay can process and transmit security-related data more quickly, allowing for faster threat detection and mitigation.
%	\item Mitigation of Denial-of-Service (DoS) Attacks: A high-performing network with enhanced throughput can better withstand DoS attacks, which seek to overwhelm the network with excessive traffic. The improved capacity of a RINA-based network can prevent such attacks from causing significant disruption.
%	\item Reliable Connectivity: Network performance directly affects the reliability of connectivity, a critical aspect of IoT networks. Reliable connectivity reduces the risk of service interruptions, which can be exploited by attackers to compromise network security.
%	\item Efficient Resource Management: RINA's superior performance enables more efficient use of network resources, reducing the opportunities for attackers to exploit resource inefficiencies and vulnerabilities.
%\end{itemize}

\subsection{Complexity Reduction}
%Considering the number of protocols, required flows, and especially required distinct mechanisms, RINA networks can satisfy security requirements with less complexity than in the current Internet. Moreover, the number of active instances of networking mechanisms is reasonably less complex in RINA with a secured link layer \cite{small2012}. Rina also reduces the size of routing tables \cite{leon2016benefits, hrizi2017hierarchical, hrizi2015sfr}.
One of RINA's salient characteristics is its ability to simplify the overall network architecture, resulting in a significant reduction in complexity. This reduction can have far-reaching implications for the security of IoT networks \cite{small2012}.

In contrast to the current Internet with its plethora of protocols, RINA networks can fulfill security requirements with fewer protocols, flows, and distinct mechanisms. By decreasing the number of active instances of networking mechanisms, the complexity of managing and securing the network is significantly reduced.

For instance, RINA networks with a secured link layer inherently have fewer active instances of networking mechanisms, leading to a more manageable and secure environment \cite{small2012}.

Furthermore, RINA reduces the size of routing tables \cite{leon2016benefits, hrizi2017hierarchical, hrizi2015sfr}, which has a dual benefit: it not only simplifies the management of the network but also minimizes the attack surface for potential intruders. Smaller routing tables mean fewer potential points of failure or compromise. The other benefits include:

\begin{itemize}
	\item \textit{Reduced Vulnerabilities}: With fewer protocols and mechanisms, there are fewer points of vulnerability that could be exploited by attackers.
	\item \textit{Easier Management}: A less complex network is easier to monitor and manage, making it simpler to detect and respond to potential security threats \cite{en14102818}. 
	\item \textit{Simpler Security Policies}: Less complexity allows for the creation of simpler, more robust security policies \cite{s20051464}. 	
	\item \textit{Better Resource Utilization}: Reducing the number of active instances of networking mechanisms allows for more efficient use of network resources, which can be beneficial for security.
\end{itemize}

\section{How RINA Addresses the Efficiency and Security Trade-off} \label{sec:rina-tradeoff}
In the evolving landscape of the Internet of Things (IoT), it has become increasingly important to address the trade-off between efficiency and security in network communications. This is where RINA comes to the fore, providing a compelling solution that does not compromise on either facet.

\subsection{Efficiency in RINA}

In terms of efficiency, RINA outperforms traditional network architectures due to its recursive and distributed nature. The architecture uses a minimalist approach, reducing the number of protocols, required flows, and distinct mechanisms, which not only simplifies network management but also leads to better resource utilization \cite{small2012}.

RINA's design also incorporates performance improvements, as demonstrated in various research studies and international projects \cite{trouva2010internet, peymanICC16}. These studies have shown that RINA effectively improves network performance in terms of throughput and delay, which are critical for IoT networks.

Moreover, the RINA model promotes a more effective Quality of Service (QoS) management \cite{gaixas2016assuring}. Connections in RINA are established based on the source's QoS requirements, which includes maximum requested bandwidth. This mechanism ensures that network resources are allocated efficiently and used optimally.

\subsection{Security in RINA}

RINA's distinct approach to network architecture also provides numerous security advantages. One of the key advantages lies in its inherent resistance to insider attacks, thanks to the wider range of control field values used, such as connection/QoS id \cite{boddapati2012assessing}.

Furthermore, the architecture's \textit{divide-and-conquer} approach reduces the overall risk posed to the network. If a DIF is compromised, it does not affect the entire network \cite{day2008networking}. Each DIF operates independently and transparently to others, providing inherent resilience and security.

Authentication is another strong security feature in RINA. All IPCPs must be authenticated before joining a DIF. This mechanism ensures that attackers cannot address IPCPs in a DIF without first undergoing the authentication process.

\subsection{Balancing Efficiency and Security}

RINA skillfully navigates the trade-off between efficiency and security in several ways. First, its programmable DIFs allow for new functionality to be developed as policies and plugged into existing mechanisms, minimizing the risk of creating new vulnerabilities while enhancing performance \cite{small2012}.

Second, its unique approach to addressing, with each DIF having its own hidden addresses, not only enhances security by making it harder for attackers to exploit IP addresses, but also increases efficiency by reducing the amount of overhead associated with managing global addresses.

Finally, RINA's simplified connection management, by adopting Watson's method, reduces the chance of connection misuse while enhancing efficiency by eliminating explicit control messages for connection establishment or closure \cite{boddapati2012assessing}.

In summary, RINA addresses the trade-off between efficiency and security by leveraging its unique recursive and distributed architecture, allowing it to provide superior performance without compromising on security. This makes RINA a promising architectural approach for the future of IoT communications.


\section{Employing RINA in IoT -- A Use Case Study}
\label{sec:usecases}

We consider two use cases to discuss how RINA can be applied on. In the first one, \textit{Home}, we assume that IoT devices can run a RINA stack, and in the second one, \textit{Health}, we assume that devices have their own legacy protocol.

\subsection{Home}
%The building \& home / smart infrastructure domain is usually covered by standard systems designed to manage building energy, environmental conditions and building security. The systems are usually provided by different vendors and can also be integrated together to provide the desired functionality. One of the major challenges here is the integration of applications that have a wide range of requirements, especially security.
%
%Fig.~\ref{fig:rina-iot} illustrates a customization of RINA for a simple, common IoT application: we assume that there are some wireless nodes connected to the network via an access point over a wireless protocol. This requires a common DIF between the two devices (with IPCPs C1 and C2). The wireless node is also a member of a DIF spanning over the network to reach the servers in the cloud. The communication, however, crosses a network segment (between C3 and C5) that is not a member of the top DIF, meaning that all the packets between B2 and B3 are (can be) out of access to Router1. This implies that the congestion controller between C3 and C5 can be customized to that segment while the congestion controller between C1 and C2 takes the wireless link into account; both path segments can have their own SDU protection policy, while B2 is able to perform any operation such as translation/conversion that is required on the PDU sent from B1 to reach B3.
The smart home scenario is an ideal testbed for the application of RINA in the IoT context. In this setting, we consider a variety of smart devices, such as temperature sensors, smart light bulbs, home security cameras, and home automation systems, all provided by different vendors. The integration of these systems can pose a challenge due to the wide range of security and operational requirements, and the variety of legacy protocols involved.

The application of RINA in this context can provide a unified networking framework to alleviate these challenges. As shown in Fig.~\ref{fig:rina-iot}, we consider a common IoT network topology where wireless nodes, such as smart home devices, connect to an access point over a wireless protocol. In this case, the communication between the two devices (with IPCPs C1 and C2) requires a common DIF. Furthermore, the wireless node must also belong to another DIF that spans over the network to reach the servers in the cloud.

In this scenario, the communication path traverses a network segment (between C3 and C5) that is not a member of the top DIF. This implies that packets between B2 and B3 are (or can be) inaccessible to Router1. This design provides an additional layer of security and privacy, as the data from the IoT devices in the home cannot be accessed or tampered with by unauthorized entities.

Moreover, this architecture allows for each segment of the path to customize its congestion control policy. For instance, the congestion controller between C3 and C5 can be tailored to the specific characteristics of that segment, while the congestion controller between C1 and C2 takes into account the conditions of the wireless link. This adaptability can enhance the efficiency and reliability of data transfers, which are key for seamless operation in a smart home environment.

Furthermore, each path segment can implement its own SDU protection policy, thereby enhancing the overall security of the network. At the same time, the node at B2 can perform any necessary operations on the PDU sent from B1 to reach B3. This feature ensures interoperability and smooth communication between IoT devices and the cloud servers.

%Overall, the application of RINA in a smart home scenario can address the key challenges of security, interoperability, and efficient resource usage, thereby providing a more seamless and secure smart home experience.



\begin{figure}[!t]
	\centering
	\includegraphics[width=0.68\linewidth]{"figures/RINA IoT".pdf}
	\caption{DIF arrangements for the home use case.}
	\label{fig:rina-iot}
\end{figure}

\subsection{Health}
%There is a need to provide timely assistance to people who run an increased risk of accidents, malfunctioning, diseases, strokes, etc., in particular elderly people, recovering or disabled people and possibly children. The goal is to monitor these people (target group) and their behavior by IoT wearable devices in order to reduce risks, to avoid accidents, to give preventive warnings, to initiate corrective measures or to ask for assistance, e.g., from nearby ``trusted'' caregivers.
%
%A sample topology of a health IoT device that does not support RINA is shown in Fig.~\ref{fig:rina-health}. As proposed by \cite{maryan2020}, legacy IoT devices can join a RINA network using a gateway called IoT sub-manager. We adopt the same approach here focusing more on the security and transport aspects of it. Referring to the figure, we observe that the network segment from IoT sub-manager to servers is the same as before; it seamlessly routes packets while considering their security without the need to an end-to-end connection. However, IoT sub-manager needs to translate the legacy protocol/packets from the IoT device, which is performed by the Trl module. Trl then passes data to A1, which is treated similarly to the data given to A1 in Fig.~\ref{fig:rina-iot}. This implies that IoT sub-manager should be a trusted device. There is another point in using IoT sub-managers: if the send rate of the IoT device is faster, i.e. there is a bottleneck in the RINA segment, Trl should be able to pushback/slow down App. This is, additionally, a part the translation Trl does.
The healthcare sector represents another significant area where IoT has been making a considerable impact, particularly in the field of patient monitoring and care. Wearable IoT devices have been instrumental in monitoring the health and behaviors of specific risk groups, such as elderly individuals, people recovering from illnesses, disabled individuals, and children. These devices aim to reduce risks, prevent accidents, issue preventive warnings, initiate corrective measures, or request assistance from \textit{trusted} caregivers when necessary.

Fig.~\ref{fig:rina-health} depicts a sample topology of a health IoT device that does not inherently support RINA. As suggested by \cite{maryan2020}, these legacy IoT devices can be incorporated into a RINA network using a special type of gateway called an IoT sub-manager. The IoT sub-manager's primary purpose is to translate a legacy (non-RINA) IoT protocol to a common RINA protocol – CDAP. CDAP achieves commonality by focusing on operations such as read/write, create/delete, and start/stop on object models. This arrangement allows for seamless integration of legacy devices while maintaining the benefits of the RINA architecture.

The network segment from the IoT sub-manager to the servers is identical to the previous scenario; it routes packets seamlessly while taking into account their security, eliminating the need for an end-to-end connection. The IoT sub-manager plays a critical role in this architecture as it is responsible for translating the legacy protocol/packets from the IoT device into a format that can be understood by the RINA network. This translation operation is performed by the Trl module within the IoT sub-manager.

Once translated, the data is passed to IPC process A1, where it is treated in the same manner as data presented to A1 in Fig.~\ref{fig:rina-iot}. This arrangement implies that the IoT sub-manager should be a trusted device due to its critical role in data translation and transmission.

An important consideration when using IoT sub-managers is managing the data send rate of the IoT device. If the IoT device transmits data faster than the RINA network segment can handle, creating a bottleneck, the Trl module within the IoT sub-manager should be able to push back or slow down the App. This capability adds to the versatility of the translation function performed by the Trl module and ensures efficient and reliable data transmission in the health IoT network.

The incorporation of legacy IoT devices into a RINA network using IoT sub-managers can significantly enhance the security, reliability, and efficiency of health IoT networks. This arrangement allows for the secure and efficient monitoring of high-risk individuals, leading to improved patient care and outcomes.


\begin{figure}[!t]
	\centering
	\includegraphics[width=0.98\linewidth]{"figures/health".pdf}
	\caption{DIF arrangements for the health use case.}
	\label{fig:rina-health}
\end{figure}


\subsection{Secure Multicast}
%In both of the above use cases, secure multicast can be performed without any challenges. Assume that the goal is to send data to several servers as shown in Fig.~\ref{fig:rina-iot} and Fig.~\ref{fig:rina-health}. This indeed does not need several separate connections from the IoT device to the servers. Instead, a multicast destination address is defined in DIF B. IPCP B1 sends the packet it gets from the IoT device to B2. Then, the RMT module in B2 checks the destination port(s); in this example, there are several servers connected to Access~1 via Router~1, each with a separate DIF with C3, C4, and C5. The RMT of B2 then copies the packet to all of these ports. As a result, IPCP C5 in all the servers receives the packet and forwards it upward. In this way, the security of the packets in DIF B is not compromised.
Secure multicast forms an integral part of IoT networking where one source node often needs to send data to multiple destination nodes. Such a feature becomes crucial in scenarios like home automation, where a single command needs to be dispatched to multiple devices, or in healthcare, where patient data may need to be shared with multiple healthcare providers simultaneously \cite{park2020security}.

In the context of RINA, secure multicast can be implemented \textit{effortlessly} and \textit{efficiently}, eliminating the need for the complex security measures described for multicast CoAP applications in \cite{park2020security}.

In RINA, security is an integral part of the architecture, not an add-on. The security in RINA is based on the principle of ``only talk to known and authenticated IPC processes''. This means that any IPCP, including those involved in multicast communication, must be authenticated before it can join a DIF. This authentication process ensures that only authorized processes can participate in the multicast communication, providing a high level of security. This also contrasts with the CoAP approach, which requires the establishment of a group key and a set of pairwise keys for secure multicast communication, a process that can be complex and resource-intensive.

Refer to Fig.\ref{fig:rina-iot} and Fig.\ref{fig:rina-health}, where the goal is to transmit data from an IoT device to several servers. This does not necessitate multiple separate connections from the IoT device to each server. Instead, RINA enables the use of a multicast destination address within DIF B. The process starts with IPCP B1 receiving a packet from the IoT device and sending it to B2. At this stage, the RMT (Relaying and Multiplexing Task) module within B2 comes into play. The RMT module examines the destination port(s) and identifies that there are multiple servers connected to Access1 via Router1, each having a distinct DIF with C3, C4, and C5.

Upon this identification, the RMT in B2 replicates the packet and dispatches it to all the identified ports. As a result, IPCP C5 in each of the servers receives a copy of the packet and forwards it to the higher layers in the stack. This multicast mechanism ensures the security of the packets within DIF B is maintained, as each packet remains within the boundaries of a single DIF, thus preserving the inherent security of the communication.

%This secure multicast mechanism provided by RINA can be highly advantageous in IoT networks, as it ensures data security while allowing for efficient and simultaneous data dissemination to multiple nodes. This feature becomes particularly important in the context of IoT, where vast numbers of devices often require simultaneous data updates, instructions, or command dispatches. By enabling secure multicast, RINA allows for efficient and secure data transmission, thereby enhancing the performance and security of IoT networks.



%In this section, we propose a privacy policy empowering RINA to manage domain synergy. Figure~\ref{fig:synergy-rina} illustrates two different IoT domains: health and home. Connecting these two on-the-fly can be performed via a new DIF with the participation of only the two nodes and the servers (with IPCPs S1 to S4). Access control policies are now customized according to the new application.

%\begin{figure}[h]
%	\centering
%	\includegraphics[width=1.05\linewidth]{"figures/RINA Domain".pdf}
%	\caption{Domain Synergy in RINA}
%	\label{fig:synergy-rina}
%\end{figure}


%Although domain synergy can be architecturally done in RINA, however, there are still some challenges such as traditional authentication and authorization methods that may not be applicable to IoT because of heterogeneity and complexity of objects. 
%
%
%
%
%
%Domain synergy and how to have a reference architecture is an important challenge in IoT. For example, we are working on such a reference architecture in the SCOTT project\footnote{http://its-wiki.no/wiki/SCOTT:Home} to foster security, reusability, scalability, and interoperability. The objectives are to leverage the future IoT design middleware mechanisms and the supporting tools needed between different industrial domains with different requirements. Based on the unification provided by RINA, a reference architecture can be conducted similarly with higher reusability and manageable policies across different domains.
%In addition, in the huge and cumbersome synergies between domains with different reference architectures, based on some work such as \cite{small2012}, RINA DIFs promote reducing the scope of networks significantly.
%However, there are still some challenges such as traditional authentication and authorization methods that may not be applicable to the IoT because of heterogeneity and complexity of objects. 
%Moreover, due to hidden addresses for applications in RINA, end-to-end authentication and authorization may encounter new issues.
%We aim to develop a lightweight and compatible structure of attribute-based access control policies \cite{ABAC13guide} for DIFs to overcome these issues.
%%
%To practically analyze how RINA is effective to prevent some existing security challenges, we are focusing on some attractive IoT devices by attackers such as CCTV cameras that are widely vulnerable to simple hacks, and we are developing an approach utilizing programmable DIFs to defeat applicable attacks to these devices.
%We are also developing metrics for measuring the security level of communication to evaluate the RINA architecture in preventing attacks.

\section{Discussion} \label{sec:disc}
\subsection{Deployment} 
%One of the main issues in the design of the IoT network stack is \textit{interoperability}, i.e. how to guarantee that IoT devices can communicate with existing Internet applications and follow Internet standards \cite{7005393}. This has made them adopt many existing protocols and apparently, inherit their vulnerabilities and design issues. 
%
%Adopting RINA, as a new protocol stack, does imply interoperability considerations which are now under investigation and implementation by some projects such as OCARINA\footnote{See http://www.mn.uio.no/ifi/english/research/projects/ocarina} that we are working on. As proposed by OCARINA and also \cite{Grasa:2012:DRP}, RINA can be deployed as an overlay/underlay/alongside other networks including the Internet; as an overlay, RINA can operate on all PHY, Link, IP, and TCP/UDP layers through its shim DIFs; as an underlay, it can seamlessly transmit, for example, TCP/IP traffic; and alongside other networks, through simple proxy IPCPs, it has been shown how RINA inter-operates with other network stacks. It has also been investigated how RINA can operate on tiny, limited devices such as wireless sensors in the RINAiSense project\footnote{See https://distrinet.cs.kuleuven.be/research/projects/RINAiSense}.
RINA represents a paradigm shift in network architecture, and like any groundbreaking technology, its adoption necessitates a phased approach. One of RINA's key strengths is its ability to operate over shim DIFs, enabling it to function even on a TCP/IP stack. This compatibility feature allows for a smoother transition from existing network architectures to RINA.

However, to fully leverage the benefits of RINA, extensive research has been conducted on its deployment strategies. For instance, the study performed in \cite{maryan2020} demonstrates how RINA can be implemented as an architecture in IoT environments where nodes are incapable of running RINA directly. In such scenarios, nodes connect to an IoT sub-manager, allowing the rest of the network to operate as a RINA network. This approach effectively integrates RINA into current IoT systems.

In \cite{ciko2019first}, the possibility of switching to RINA was evaluated in case a node can join a DIF directly via its first connected hop. In the RINAiSense project\footnote{See \url{https://distrinet.cs.kuleuven.be/research/projects/RINAiSense}}, the possibility of running RINA on limited devices such as sensors has been investigated. Deployment possibilities of RINA is also one of the main goals of OCARINA\footnote{See \url{http://www.mn.uio.no/ifi/english/research/projects/ocarina}}. It was also shown that legacy applications that can use the TAPS API, can use RINA via a mapping of TAPS to RINA \cite{krist2020taps}.


%\section{Related Work}
%\label{sec:relatedwork}
%There are many published surveys on IoT security issues and challenges. Granjal et al. [6] analyzed existing solutions for the IoT standardized communication protocols (PHY, MAC, Network, Application) and cross-layer mechanisms whenever applicable. Sicari et al. [7] presented research challenges and the current solutions in the field of IoT security focusing on the main security issues which were identified in seven categories: authentication, access control, confidentiality, privacy, trust, secure middleware, mobile security, and policy enforcement.
%
%Roman et al. [8] focused on the analysis of the centralized and distributed approaches. They introduced an attacker model that was applied to both centralized and distributed IoT architectures, and studied the main challenges and promising solutions in the design and deployment of the security mechanisms. \red{May need to check!}
%
%%\red{delete this?!} In this survey paper, we explore the IoT security and privacy issues in four aspects. The first part presents the most relevant limitations of IoT devices and their solutions. The second part discusses the classification of existing IoT attacks. Then, we \cite{yang2017survey} explors the IoT authentication and access control schemes and architectures proposed in recent literature. Finally, we analyze the security issues and mechanisms in the perception layer, network layer, transport layer, and application layer, respectively.\red{We may show how they can be prevented by RINA}
%
%There is also another type of research work (e.g. \cite{SUAREZ2016190,6924301}) which focuses on presenting a secure IoT architecture. The presented architecture usually operates at higher layers regardless of what the network stack is. On the contrary, in this paper, we fundamentally looked at the security issues of lower layers, and especially the network stack and its protocols.
%

\subsection{Further Research Topics}
The potential of RINA to facilitate secure end-to-end connections while allowing for their segmentation presents intriguing research opportunities and challenges. A key consideration in this context is performance optimization, particularly the implementation of an effective feedback mechanism within the recursive architecture. This mechanism is crucial for managing congestion controllers in a sequenced or stacked configuration \cite{peymanICC16}.


In our previous work \cite{hayes2016feedback}, we conducted an analytical evaluation of various feedback methods, assessing their impact on system stability and average queue length. Our findings indicated that strict pushback feedback, based solely on queue size, could lead to stability issues. This observation as well as our study in \cite{welzl2022future} underscore the need for a more sophisticated feedback method that does not rely exclusively on queue size.

In the same study \cite{hayes2016feedback}, we utilized Scalable TCP. However, we believe that the performance could be further enhanced by adopting a congestion controller such as LGC \cite{teymoori2016even, teymoori2020lgcc, ciko2022lgc}. In LGC, we employed the same ECN signaling used by DCTCP, but with a lower marking threshold. This modification resulted in smoother behavior and a shorter queue length.

Currently, we are developing a multihop congestion controller based on LGC for RINA. The outline of a recursive congestion control mechanism is depicted in \cite{welzl2020follow}. This controller can leverage the flow aggregation capability provided by lower DIFs. This feature allows IoT devices to perform data aggregation \cite{teymoori2012real}, which can significantly reduce energy consumption. This is in contrast to packet aggregation \cite{6478424}, which is performed at the physical (or lower) layer to decrease protocol overhead \cite{5734075}. Interestingly, these two approaches can be employed simultaneously in RINA, offering potential for enhanced efficiency.

In the context of IoT, the ability to perform both data and packet aggregation can have significant implications. Data aggregation \cite{teymoori2012real} allows IoT devices to consolidate and summarize data before transmission, reducing the amount of data that needs to be transmitted and thus saving energy. This is particularly important for battery-powered IoT devices, where energy efficiency is a key concern.

On the other hand, packet aggregation \cite{6478424} is performed at the physical (or lower) layer to decrease protocol overhead \cite{5734075}. By combining multiple smaller packets into a larger one, packet aggregation can reduce the per-packet overhead, leading to more efficient use of network resources.

Furthermore, the development of appropriate SDU Protection policies is another important research direction. SDU Protection policies in RINA can provide a range of security services, including data integrity, confidentiality, and authentication. Developing SDU Protection policies that are tailored for constrained IoT devices can help to ensure that these devices can securely participate in a RINA network, despite their limited resources.

Finally, the development of multicast protocols for constrained IoT devices is another interesting research challenge. Multicast communication can be more efficient than unicast communication for certain types of IoT applications, such as those that involve group communication or data dissemination. However, designing multicast protocols that can operate efficiently and securely on constrained IoT devices is a non-trivial task.

\section{Conclusion}
\label{sec:conclusion}
%In this paper, we discussed some main architectural performance and security issues of IoT networks. We showed the challenges in current IoT network stacks and why achieving a secure transport is a trade-off. Then, we discussed the recursive idea behind RINA, a promising network architecture, and explained its main modules for data transfer. Combined with embedded security at each layer, we demonstrated how transport layer security does not and should not architecturally contradict efficiency. In each layer, the EFCP module of RINA provides features hardening inside attackers. Between layers, security is ensured by SDU protection; separating these two allows any protocol translation / connection split in the path.
%
%We presented two use cases that can benefit from our proposed solution. In both cases that an IoT device can run the RINA stack locally or not, we illustrated architectural solutions, which also facilitates performing a secure multicast; this has been a serious demand in the IoT, but without any support by current protocols. We also discussed different deployment methods for RINA and our future directions.
This paper delved into the inherent architectural performance and security intricacies of IoT networks, shedding light on the numerous challenges present in the current IoT network stacks. We illustrated that securing transport layer communication is not a straightforward task but a delicate trade-off balancing performance and security in IoT.

In the pursuit of an effective solution, we turned our attention towards RINA, a promising network architecture with recursive design principles. We explicated the fundamental data transfer mechanisms within RINA and emphasized the importance of embedded security at each layer of the communication stack.

Our exploration revealed that RINA's design inherently reconciles the traditional trade-off between transport layer security and efficiency. We showed that the EFCP module within each layer of RINA offers features that significantly resist against internal attackers. Between layers, RINA ensures security through SDU protection, separating this intra-layer and inter-layer security, which, in turn, facilitates protocol translation and connection splitting along the communication path.

To illustrate the practical application of our proposed solution, we presented two distinctive use cases, each benefiting uniquely from the integration of RINA. Whether an IoT device natively supports the RINA stack or not, we demonstrated architectural solutions that not only enhance security and performance but also facilitate secure multicast. Notably, secure multicast has been a long-standing requirement in the IoT landscape, yet remains largely unaddressed by current protocols while considering efficiency as well.

Our discussion extended to various RINA deployment methods, providing a comprehensive understanding of how this novel architecture can be practically implemented within diverse IoT scenarios. As we continue our exploration in this field, we envision future directions that further push the boundaries of IoT network performance and security, building upon the foundational principles and methods established in this paper.

In conclusion, we believe that RINA represents a significant stride towards overcoming the challenges that IoT networks face today. By rethinking network architecture and leveraging recursion, RINA provides a robust and scalable solution that stands to redefine the security and efficiency paradigms in IoT networks.

\section*{Acknowledgment}
It should also be noted that while the content of this paper is human-generated, Generative-AI, in particular ChatGPT-4 (\underline{https://chat.openai.com/}) has been used for editing, proof-reading, and increasing the readability of sentences. An additional round of proof-reading by the authors has been done in the end.

\Urlmuskip=0mu plus 1mu\relax

\bibliographystyle{IEEEtran}
% argument is your BibTeX string definitions and bibliography database(s)
%\balance
\bibliography{IEEEabrv,Bibfile}

\begin{IEEEbiography}[{\includegraphics[width=1in,height=1.25in,clip,keepaspectratio]{figures/pteym.jpg}}]{Peyman Teymoori} is an Associate Professor at the University of South-Eastern Norway (USN). He obtained his Ph.D. in Computer Science from the University of Tehran, specializing in Wireless Adhoc Networks. Post-graduation, he further enriched his research portfolio as a Postdoctoral and then as a Senior Research Fellow at the University of Oslo, before transitioning to his current role at USN. His research area includes the modeling, optimization, and performance evaluation of communication networks, the Recursive InterNetwork Architecture (RINA), and networking technologies like Internet of Things (IoT), 5G/6G, WiFi, and Ad hoc Networks.
\end{IEEEbiography}

\begin{IEEEbiography}[{\includegraphics[width=1in,height=1.25in,clip,keepaspectratio]{figures/toktam.jpg}}]{Toktam Ramezanifarkhani}  is an Associate Professor specializing in Cybersecurity. She has devoted her academic and professional career to the field of information security, with a particular focus on various aspects of cybersecurity. Her research interests span a broad spectrum, ranging from theoretical foundations to practical applications. These include software security, application and language-based security, vulnerability analysis, penetration testing, IoT security, and the application of formal methods in information security. She also has a keen interest in the human aspects of cybersecurity, recognizing the critical role that individuals play in maintaining secure systems.
\end{IEEEbiography}
\balance

\EOD

\end{document}
